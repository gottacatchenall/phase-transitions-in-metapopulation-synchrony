\hypertarget{phase-transitions-in-metapopulation-synchrony}{%
\section{Phase Transitions in Metapopulation
Synchrony}\label{phase-transitions-in-metapopulation-synchrony}}

\begin{quote}
\begin{flushright}Nature is a mutable cloud, which is always and never the same.

Ralph Waldo Emerson \end{flushright}
\end{quote}

\hypertarget{introduction}{%
\subsection{Introduction}\label{introduction}}

Ecosystems are quintessential complex systems. Ecological processes are
inherently the product of interactions across all scales of biological
organization, from the interactions between electrons that drive
biochemical processes, to interactions between individual cells that
constitute multicellular life, to the interactions between separate
organisms in population ecology, to interactions between species in
community ecology, to interactions between biogeographical patterns and
biosphere level forces like climate (@levin\_problem\_1992). This
process of \emph{emergence}, by which parts come together to form a
whole with properties that don't exist among the individual parts, has
been studied across a wide variety of disciplines
(@manrubia\_emergence\_2004), and is a ubiquitous phenomenon in complex
systems.

One potential cause of emergent behavior in complex systems is
\emph{synchrony} among individual parts. When many independent parts
come together to act as a whole, their dynamics become
\emph{synchronized}. This behavior is ubiquitous in biological systems
across all scale of organization. From collections of cells acting
together: the heart beating in rhythm (@womelsdorf\_modulation\_2007),
neurons firing in unison (@strogatz\_sync\_2003)---to behavior among
organisms: the flash of fireflies (@otte\_theories\_1980) or the
migration of birds (@spottiswoode\_extrapair\_2004)---to interactions
between organisms: synchrony between abundances of predators and prey,
and of phenology (@van\_asch\_phenology\_2007, @burkle\_future\_2011).
Synchrony, by definition, involves different entities changing over time
in the same way. Within ecology, there has long been a focus on spatial
synchrony, that is---how does spatial distribution of ecological
entities affect whether they change together or separately?
(@jarillo\_spatial\_2020, @kendall\_dispersal\_2000,
@hanski\_spatial\_1993). This is, in large part, due to the applied
importance of understanding the effect of habitat loss on natural
populations. Many theoretical studies have shown the two primary factors
that develop spatial synchrony across space are dispersal and
environmental covariance (@ripa\_analysing\_2000, @abbott\_does\_2007).
Within this theory, one maxim that has developed is \emph{Moran's rule},
which states that spatial synchrony is proportional to the covariance in
the environmental conditions across space (@ranta\_synchrony\_1995,
@bjornstad\_spatial\_1999).

A major goal of conservation has been developing corridors to promote
landscape connectivity. Most measures of landscape connectivity
represent \emph{structural} connectivity, meaning quantifying the
structure of the landscape, rather than \emph{functional} connectivity,
which measures the connectivity of a given process
(@kool\_population\_2013, @calabrese\_comparison-shoppers\_2004). Here,
in order to better understand functional connectivity, we use a
simulation model to measure how synchrony across space changes as a
function of landscape structure. We do this by developing a model of
metapopulation dynamics on spatial graphs, which have long been used to
model landscape connectivity (@martensen\_spatio-temporal\_2017,
@albert\_applying\_2017, @urban\_landscape\_2001), and analyze how
synchrony changes across space using the language of critical
transitions.

Further, we show that increasing population synchrony reduces the
variance in the generation-to-generation change in abundance, which is
central in reducing the probability of metapopulation extinction
(@lande\_risks\_1993, @lande\_extinction\_1998). This relationship
suggests that promoting functional landscape connectivity can help
mediate the probability of extinction for species facing significant
habitat loss. We suggest using simulation models, such as those
presented here, to aid in decision making regarding corridor placement.

\hypertarget{what-are-phase-transitions}{%
\subsection{What are phase
transitions?}\label{what-are-phase-transitions}}

When does a system change from one state to a different state? Due to
the rapid changes induced on the planet by human activity, there has
been recent focus on answering this question in ecology, especially the
potential for changes in spatial structure to drive transitions between
alternative stable states. Much of this theory has been aimed at the
practical problem of being able to predicting the onset of transitions
from time-series data (@scheffer\_anticipating\_2012,
@scheffer\_early-warning\_2009). The bulk of the quantitative theory
used to understand transitions between states is derived from
statistical mechanics, where it was originally used to study
phase-transitions in matter. In order to study phase transitions between
regimes, we must first be clear on what these regimes are. As this
theory was originally used to describe physical states of matter, the
original regimes were solid, liquid, gas. However, as our understanding
of condensed-matter has changed, so has the demarcation of what
constitutes different states. In reality, the way in which particles
come together to constitute matter is far more variable than these three
categories. In such cases the `state' of a collection of particles
cannot be represented by a single categorical label, and so the theory
of phase transitions as been adapted to model \emph{continuous} phase
transitions, where there is no clear demarcation point between different
states (@sethna\_statistical\_2006). This is useful for us in ecology,
where the line between alternative ecosystem `states' is even fuzzier.

We can formalize our understanding of phase transitions using the
language of statistical mechanics. We call the \emph{order parameter}
some measure of the system's state in space and time. The \emph{control
parameter}, then, is what causes the change in order parameter. When
dealing with dynamics that are inherently stochastic, one tool often
used in statistical mechanics is correlation functions which measure how
well the order parameter is correlated in both space and time at a
particular value of the control parameter (@sethna\_statistical\_2006).
For example, if we consider the population of a species inhabiting a
landscape, where along the gradient of landscape connectivity, our
control parameter, does that system go from consisting of one large,
single, population, to many small, independent populations? We measure
this qualitative shift from one system to many using \emph{synchrony},
the correlation in the dynamics of abundance across space.

\pagebreak

\hypertarget{the-model}{%
\subsection{The Model}\label{the-model}}

Here we present a spatial graph model of landscape connectivity based on
metapopulation theory (@hanski\_practical\_1994,
@grilli\_metapopulation\_2015) . We model connectivity as a function of
a few empirically estimable parameters, and then describe a diffusion
model of metapopulation dynamics on these spatial graphs. We then
simulate dynamics across parameters representative of landscape
connectivity to determine where transitions in the synchrony of
population dynamics across space occur.

\hypertarget{modeling-landscape-connectivity-with-a-spatial-graph}{%
\subsubsection{Modeling Landscape Connectivity with a Spatial
Graph}\label{modeling-landscape-connectivity-with-a-spatial-graph}}

Spatial graphs have long been used to model a system of habitat patches
(@dale\_graphs\_2010, @minor\_graph-theory\_2008,
@urban\_landscape\_2001). In figure \ref{concept}, we see a conceptual
example of how these graphs are constructed from land cover data.

\begin{figure}[h]

\includegraphics[width=15cm]{/Users/michael/Drive/thesis/figures/concept.png}

\caption{Spatial graph representation of a heterogenous landscape. Panel (a): A raster map of landscape classification, with each land cover type denoted in a different color. Panel (b): The habitat type of interest (green), repreented as a set of contiguous patches. Panel (c): Each patch represented as a node at the patch's centroid. Panel (d): A spatial graph derived from the landscape. }

\label{concept}

\end{figure}

Here we model a system of populations, represented as a vector of
vertices, \(V\), in a spatial graph \(G=(V,E)\), with edges, \(E\),
representing dispersal between populations. When considering the process
of metapopulation dynamics, we choose to model landscape connectivity as
a combination of two different factors: the probability than any
individual migrates during its lifetime, \(p_m\), and the conditional
distribution over spatial nodes of where an individual goes, given that
it migrates, \(P(V_j|V_i)\), which we call the dispersal potential.

We can model the dispersal potential using a few methods. In empirical
systems, this can be estimated with resistance surfaces, which provide
relative weights of the difficulty of migration between points on a
raster of land-cover type (@spear\_use\_2010). Theoretically, we model
the dispersal potential using isolation-by-distance (IBD). The relative
probability of dispersal between \(V_i\) to \(V_j\) is inversely
proportional to the distance between them, \(d_{ij}\), and the strength
of the isolation-by-distance relationship, \(\alpha\). We call the
functional form of this relationship, \(f(d_{ij}, \alpha)\), the
dispersal kernel. Here we consider two different types (partially after
Grilli et al.~(2015)), the
exponential, \(f(d_{ij}, \alpha)=e^{-\alpha d_{ij}}\), and
Gaussian, \(f(d_{ij}, \alpha)=e^{-\alpha^2 d_{ij}^2}\), kernels. These
functional forms have long been considered as models of dispersal
kernels in both theory and empirical work (@hanski\_practical\_1994,
@grilli\_metapopulation\_2015).

To convert this to a probability distribution \(P(V_j|V_i)\), we
normalize:

\[P(V_j|V_i)=\frac{f(d_{ij}, \alpha)}{\sum_k f(d_{ik},\alpha)}\]

Note that if \(\alpha=0\), then the value of both exponential and
Gaussian kernels is the same for all pairs of populations, and therefore
the dispersal potential is a uniform distribution. In figure \ref{mp},
we can see spatial graphs plotted representing the same set of
populations across differing values.

\begin{figure}[h]

\includegraphics[width=15cm]{/Users/michael/Drive/thesis/figures/mp.png}

\caption{The same set of populations plotted in the x-y plane, represented as blue points, across various value of isolation-by-distance strength, represented by (columns), and dispersal kernel type (rows). Edge width and opacity are proportional to the dispersal kernel between those two populations, $P(V_j | V_i)$.}

\label{mp}

\end{figure}

\hypertarget{local-dynamics}{%
\subsubsection{Local Dynamics}\label{local-dynamics}}

We model population dynamics within each local population \(V_i\) using
the stochastic logistic model. The dynamics of the number of individuals
in population \(V_i\) are described by the stochastic differential
equation (SDE)

\[dN_i=\lambda_i N_i \big(1 - \frac{N_i}{K_i}\big)dt+\sigma_pK_idW\]

Here, \(N_i\) is the abundance at population \(i\), \(\lambda_i\) is the
strength of density dependence in that population, and \(K_i\) is the
carrying capacity of \(V_i\). \(\sigma_p\) represents that standard
deviation in abundance due to local stochasticity as a proportion of
\(K_i\). Here \(\sigma_p\) represents an amalgamation of all factors
contributing to local stochasticity in population dynamics, although it
should be noted that the relative contribution of different factors to
local stochasticity can drive significant variation in dynamics
(@melbourne\_extinction\_2008). For the sake of reducing parameter
space, here we consider all populations as having the same \(\lambda_i\)
and \(K_i\), however, future work could include exploring the
source-sink dynamics in this system by varying intrinsic growth rates
and carrying capacities across populations.

We use an SDE representation because they have been used to study phase
transitions in stochastic systems before (@brock\_early\_2012,
@kuehn\_mathematical\_2011), and they have many nice properties. SDEs
have been used to study extinction dynamics before, as it is relatively
straightforward to compute the mean time until extinction (MTE) using
the Kolmogorov Backward Equation (@lande\_stochastic\_2003).

\hypertarget{diffusion-on-spatial-graphs}{%
\subsubsection{Diffusion on Spatial
Graphs}\label{diffusion-on-spatial-graphs}}

When we model a landscape with a spatial graph, we have to decide how
different nodes affect one another. The processes that connect
landscapes are inherently stochastic. The probability that an individual
migrates within its lifetime, \(p_m\), and where it goes, \(P(V_j|V_i)\)
, are both stochastic processes. Under some conditions, we can
effectively model stochastic dynamics across space using diffusion
models. In essence, a diffusion model assumes that at each timestep the
system will change according to the expected value of stochastic
dispersal. Diffusion models have seen widespread use in ecology and
other fields (@ovaskainen\_empirical\_2008,
@holmes\_partial-differential\_1994, @okubo\_diffusion\_2011).

Given our dispersal potential \(P(V_j|V_i)\), we can define the
diffusion matrix \(\Phi\) as

\[\Phi_{ij} = \begin{cases} 1 - p_m \quad\quad &\text{if}\ i = j \\ P(V_j|V_i)p_m & \text{if}\ i \neq j\end{cases}\]

where \(\Phi_{ij}\) represents the probability that any individual born
in \(V_i\) reproduces in \(V_j\).

Now we move to considering the dynamics of the abundance of the
population at \(V_i\), denoted \(N_i\). We can now represent the
dynamics due to diffusion of this system as

\[\dot{N_i}=(1-p_m)N_i+ \sum_j p_mP(V_j|V_i)N_j\]

In matrix notation, we can represent this diffusion model as

\[\frac{d\vec{N}}{dt}=\Phi^T\vec{N}\]

We can then combine this with local dynamics as a reaction-diffusion
model,

\[\frac{d\vec{N}}{dt} = g(\Phi^T \vec{N})\]

where \(g(x)\) is a function that represents the hypothesized mechanism
of how the ecological measurement evolves locally.

In principle, \(g(x)\) can represent any ecological process of
interest---for example if the state space of \(x\) is allelic
frequencies, \(g(x)\) could describe genetic drift, or if \(x\)
represents community compositions across space, \(g(x)\) could describe
competition between species as a function of environmental conditions,
coevolutionary states across space, etc. Here, we consider \(g(x)\) to
be the stochastic logistic model (see previous section). Combining this
with the diffusion model yields the SDE

\begin{equation} \label{diffusion_model}
  d \vec{N} = {I} (\Phi^T \vec{N})(\vec{k}-\vec{N}) dt + \sigma_p \vec{k} \ d\vec{W}
\end{equation}

which will be the primary object of study in this paper.

\hypertarget{measuring-synchrony}{%
\subsubsection{Measuring Synchrony}\label{measuring-synchrony}}

As described above, measures of correlation in space and time are often
used in the study of phase transitions. When the order parameter gains
or loses correlation in space and time is used as an indicator of when a
system changes qualitative phases. Within the context of ecology, the
crosscorrelation function, \(CC\), has long been used as a measure of
synchronous dynamics (@liebhold\_spatial\_2004). Here, with a subdivided
population, we consider the mean crosscorrelation compared across all
populations, \({PCC}=\frac{1}{n^2}\sum_{i,j} CC(V_i,V_j)\) . As an
example, in \ref{async_and_sync}, panel (a) show an example of low PCC,
and panel (b) shows an example of high PCC.

\begin{figure}[h]

\includegraphics[width=15cm]{/Users/michael/Drive/thesis/figures/async_and_sync.png}

\caption{The abundance of 5 populations (each in a different color) across time. Panel (a):  These populations show low synchrony, $PCC=0.1$. Panel (b): These populations show high sychrony, $PCC=0.89$ }

\label{async_and_sync}

\end{figure}

\hypertarget{simulating-a-phase-transition-across-a-migration-gradient}{%
\subsubsection{Simulating a Phase Transition across a Migration
Gradient}\label{simulating-a-phase-transition-across-a-migration-gradient}}

Now we consider how synchrony changes as a function of \(p_m\). Each run
consists of the following parameters,
\(\theta = \{N_p, \lambda, \sigma_p, \vec{K},p_m, \alpha \}\). For each
unique set of parameters, we run \(50\) replicates across all values of
\(p_m = \{0.01,0.02,\dots,1.0 \}\). For each replicate, we independently
draw the location of \(N_p\) populations uniformly in \([0,1]^2\). We
draw the intial value of abundance for all populations from a uniform
distribution, \(N_i \sim U(0, K_i)\). We then integrate equation
\ref{diffusion_model} forward \(500\) timesteps using the
Euler--Maruyama method with \(\Delta t=0.1\). After integrating forward,
we compute the crosscorrelation coefficient, \(CC\) for each pair of
populations \(i \neq j\). Then, we compute the mean pairwise
crosscorrelation for that replicate, \(PCC\), and start the next
replicate.

\pagebreak

\hypertarget{phase-transitions-in-synchrony-across-a-migration-gradient}{%
\subsection{Phase Transitions in Synchrony across a Migration
Gradient}\label{phase-transitions-in-synchrony-across-a-migration-gradient}}

\hypertarget{synchrony-phase-diagrams}{%
\subsubsection{Synchrony Phase
Diagrams}\label{synchrony-phase-diagrams}}

Here, we introduce synchrony transition diagrams. When does the
equilibrium state of our order parameter, \(\text{PCC}\), change as we
change our control parameter, \(p_m\)? First, we consider the case of
only two populations, \(N_p=2\). Here, the dispersal potential,
\(P(V_j|V_i)\), becomes irrelevant, as the fact that an individual
migrates means that it migrates to the other
population---i.e.~\(P(L_2|L_1)=1\) and \(P(L_1|L_2)=1\). Here, then,
\(p_m\) is the only parameter driving connectivity in the system. How
does the expected amount of synchrony between the two populations change
as we change the probability an individual migrates, \(p_m\)? In Figure
\ref{2pops_across_lambda}, we see \(\text{PCC}\) plotted as a function
of \(p_m\). Each point represents the mean value of \(\text{PCC}\)
across 50 replicates. Each panel of figure \ref{2pops_across_lambda} is
for different strengths of density-dependent regulation, \(\lambda\).

\begin{figure}[h]

\includegraphics[width=15cm]{/Users/michael/Drive/thesis/figures/2pops_across_lambda.png}

\caption{Each panel is a synchrony transition diagram for differing values of density-dependent regulation, $\lambda$. Each panel considers how mean $PCC$ changes as a function of the probability of migration, $p_m$. Each point represents the mean value of $PCC$ across 50 replicates at that parameter set.}

\label{2pops_across_lambda}

\end{figure}

Each plot in figure \ref{2pops_across_lambda} is a phase transition
diagram, showing how synchrony changes as a function of migration
probability. Each show a similar functional form, as we see correlation
between the two populations maximized near \(p_m=0.5\). Intuitively,
this maximum makes sense: exchanging 50\% of the individuals between
populations in each time step leads to a well-mixed metapopulation.
Furthermore, since there is no local adaptation in this model, a
migration rate of \(p\) is functionally equivalent to a migration rate
of \(1 - p\).

We see that this is a continuous, or type-II phase transition---we don't
reach a point of \(p_m\) where the system suddenly jumps from
\(\text{PCC}= 0\) to \(\text{PCC}=1\). Instead, we see that the expected
level of synchrony drops off as we move away from \(p_m=0.5\). Further,
as we increase \(\lambda\), we see this peak become `narrower', in that
values near \(p_m = 0.5\) have lower expected \(PCC\) as \(\lambda\)
gets larger. Why does increasing the strength of density dependent
regulation, \(\lambda\), narrow the curve? As \(\lambda\) increases,
values of \(p_m\) that enable in-between levels of synchrony for low
\(\lambda\) have their moderate synchrony broken down by local
density-dependent regulation, unless the populations are at a value of
\(p_m\) such that they synchronized tightly enough that they are both
experiencing the same force of density-dependent regulation, meaning
moving either up or down towards \(K_i\), at the same time.

In figure \ref{2pops_across_sigma}, we see transition diagrams for
\(N_p=2\) across across demographic stochasticity, \(\sigma_p\) .

\begin{figure}[h]

\includegraphics[width=15cm]{/Users/michael/Drive/thesis/figures/2pops_across_sigma.png}

\caption{Each panel is a synchrony transition diagram for differing values of density-dependent regulation, $\sigma$. Each panel considers how mean $PCC$ changes as a function of the probability of migration, $p_m$. Each point represents the mean value of $PCC$ across 50 replicates at that parameter set.}

\label{2pops_across_sigma}

\end{figure}

Here, we see that amount of demographic stochasticity doesn't
dramatically alter the shape of the synchrony transition. This leaves us
with a few questions---why would this phase transition by centered
around \(p_m = 0.5\)? Here we suggest the following hypothesis: at
\(p_m=0.5\), the composition of each population at any time is equally
composed of descendants from each population---this enables even-mixing
of demographic stochasticity within each population, which maximizes
synchrony. We make this case further in \emph{Synchrony due to
Even-Mixing of Local Stochasticity}. Why does the strength of
density-dependence, \(\lambda\), narrow the region of \(p_m\) where
synchrony is maximized, but \(\sigma_p\) doesn't? We suggest that this
is due to the small amount of subdivision present, which we'll examine
in the next section.

\hypertarget{effects-of-demography-and-landscape-connectivity}{%
\subsubsection{Effects of Demography and Landscape
Connectivity}\label{effects-of-demography-and-landscape-connectivity}}

Now, we vary demographic parameters, \(N_p\), \(K\), \(\sigma_p\) , and
\(\lambda\) , to determine how shifts in demography change the
functional form of the transition across \(p_m\). In figure
\ref{subdiv_across_lambda}, we vary the strength of density dependence,
\(\lambda\), across different numbers of populations, \(N_p\).

\begin{figure}[h]

\includegraphics[width=15cm]{/Users/michael/Drive/thesis/figures/subdiv_across_lambda.png}

\caption{Each panel is a transition diagram showing $PCC$ across $p_m$ at different combinations of $\lambda$ and $N_p$.}

\label{subdiv_across_lambda}

\end{figure}

Here we see that both \(\lambda\) and \(N_p\) have effects on the
structure of the curve. Like in the two-population model, increasing
\(\lambda\) narrows the peak around which \(p_m\) values produce
synchrony. Let \(m_{max}\) be the value of migration that maximizes
synchrony. We see that as subdivision, \(N_p\), increases, we see the
transition curve narrows around increasingly high values of \(m_{max}\).
This supports with our hypothesis of even-mixing, which we'll explore
more later. In figure \ref{subdiv_across_lambda}, each color represents
a different value of isolation-by-distance strength, \(\alpha\). We see
that changing \(\alpha\) doesn't change the shape of the curve, but
instead just flattens it---decreasing the maximum value of synchrony
across the \(p_m\) gradient. This is intuitive---as connectivity
decreases, it becomes less likely to produce global synchrony at any
\(p_m\) .

Next we varied the total carrying capacity, \(K\), across different
numbers of populations, \(n_p\) (Figure \ref{subdiv_across_k}).

\begin{figure}[h]

\includegraphics[width=15cm]{/Users/michael/Drive/thesis/figures/subdiv_across_k.png}

\caption{Each panel is a transition diagram showing $PCC$ across $p_m$ at different combinations of $K$ and $N_p$.}\label{subdiv_across_k}

\end{figure}

In Figure \ref{subdiv_across_k}, the number of populations has the same
effect as above, but carrying capacity doesn't alter the shape of this
curve, which is unsurprising given that our model considers the variance
of local stochasticity to be a proportion of \(K\).

Finally, we consider shifting \(\sigma_p\). (Figure
\ref{subdiv_across_sigma}). Here, we see that the magnitude of
\(\sigma_p\) narrows the curve, as larger amounts of local stochasticity
drown-out synchrony at values near, but not at, \(m_{max}\). Here we
observe the same effect of subdivision, and similar narrowing around
\(m_{max}\) . Unlike in the two-population model, more variability
locally does mean less synchrony at same migration levels. We suggest
that this is because increasing subdivision increases the number of
stochastic demographic events that happen every timestep, which
inherently increases the variance in the change in abundances during any
interval.

\begin{figure}[h]

\includegraphics[width=15cm]{/Users/michael/Drive/thesis/figures/subdiv_across_sigma.png}

\caption{Each panel is a transition diagram showing $PCC$ across $p_m$ at different combinations of $\sigma_p$ and $N_p$.}\label{subdiv_across_sigma}

\end{figure}

Clearly, we see landscape subdivision, represented by \(N_p\), as having
the most prominent effect on the shape of the transition curve. The more
landscape subdivision there is, the higher migration has to be to enable
synchrony. Further, subdivision induces a qualitative shift in the shape
of this curve, from being only convex, (e.g., figure
\ref{subdiv_across_sigma}A), to having both concave and convex regions
(e.g., figure \ref{subdiv_across_sigma}D). As subdivision increases, the
transition curves exhibit more `critical' behavior, meaning they reach a
tipping point value of \(p_m\) where synchrony starts to increase very
suddenly (figure \ref{subdiv_across_sigma}(A-D)).

We next considered how \(\text{PCC}\) changes across both \(p_m\) and
\(\alpha\) with different dispersal kernels. Figure \ref{kernels} shows
that the dispersal kernel has significant effects on the region where
global synchrony is established. This means that different structures of
resistance features lead to different outcomes in the ability to build
up global synchrony.

\begin{figure}[h]

\includegraphics[width=15cm]{/Users/michael/Drive/thesis/figures/kernels.png}

\caption{Phase digram showing $PCC$ across $\alpha$  (x-axis) and $p_m$ (y-axis) for the exponential dispersal kernel (left) and gaussian dispersal kernel (right). }

\label{kernels}

\end{figure}

\hypertarget{synchrony-due-to-even-mixing-of-local-stochasticity}{%
\subsection{Synchrony due to Even-Mixing of Local
Stochasticity}\label{synchrony-due-to-even-mixing-of-local-stochasticity}}

The Moran effect claims that the synchrony of populations across space
in proportional to the covariance of their environmental conditions
(@ranta\_synchrony\_1995, @bjornstad\_spatial\_1999). Here we make the
case that what drives synchrony across space is the `even-mixing' of
local stochasticity. As is evidenced by figures \ref{2pops_across_sigma}
and \ref{2pops_across_lambda}, in a two population scenario, synchrony
is maximized near \(p_m =0.5\). We suggest that this is because the
population at any given time step is an even mix of descendants from
each population, and thereby averages out the demographic stochasticity
across all populations, leading to synchronized dynamics, as the change
in each population at every time is the averaged value of local change
across all populations. If this were the case, we would expect the value
of migration which maximizes synchrony,
\(m_{max}=arg max_{p_{m}} PCC(p_m)\) to be the value such that the
probability of an individual reproduces in population \(j\) given they
were born in population \(i\), \(\Phi_{ij}\), to be equal for all \(j\).
In the case that \(P(V_j | V_i)\) is uniform, i.e.~\(\alpha = 0\), this
would mean setting \(p_m = m_{max}=1 − \frac{1}{N_p}\) would result in a
\(\Phi_{ij}\) that evenly mixes indivudals each timestep. In figure
\ref{even_mixing}, we can see this function is plotted in black against
simulated data points (purple).

\begin{figure}[h]

\includegraphics[width=15cm]{/Users/michael/Drive/thesis/figures/even_mixing.png}

\caption{Each panel represents number of populations, $N_p$ on the x-axis, and $p_m$  on the y-axis. The value of $p_m$ which maximizes syncrhony across a $p_m$ gradient, with $50$ replicates at each value of $p_m$, is represented in purple. The line in black represents even mixing,  $m_{max}=1 - \frac{1}{N_p}$. Each panel represents a different strength of isolation-by-distance, $\alpha$.}

\label{even_mixing}

\end{figure}

This indicates that synchrony is enabled when local stochasticity is
averaged out by dispersal. The Moran effect suggests that population
synchrony across space is proportional to covariance in environment.
However, dispersal can substitute for covariance in the environment when
it results in the admixture of individuals from populations in
proportion to the environmental covariance. Further, we see in
\ref{even_mixing} that increasing the strength of isolation-by-distance
this relationship weaker, as different subsets of the total set of
populations form synchronized dynamics (see the final section,
\emph{Mesopopulation}).

\hypertarget{synchrony-and-variance}{%
\subsubsection{Synchrony and Variance}\label{synchrony-and-variance}}

Consider the change in size of population \(i\) size across all
timesteps, \(\Delta_i = [\Delta_i(t) - \Delta_i(t-1)] \ \forall \ t\).
How does the global synchrony of the metapopulation change the variance
of this distribution?

In figure \ref{variance_over_m}, we see how the variance of \(\Delta\),
\(V(\Delta)\), changes with \(p_m\). We see that the overall variance in
the change in abundance over time is reduced as \(p_m\) increases, in a
pattern strikingly similar to the way \(PCC\) changes over \(p_m\). In
panel (b), we see that \(V(\Delta)\) is negatively correlated with
\(PCC\), and that this relationship becomes stronger as \(\sigma_p\)
increases. This indicates that synchrony reduces the values of
\(V(\Delta)\). Now consider the value of \(p_m\) which minimizes
\(V(\Delta)\). How correlated is this with the value of \(p_m\) that
maximizes synchrony, \(m_{crit}\) ? Consider figure
\ref{delta_pcc_corr}.

\begin{figure}[h]

\includegraphics[width=15cm]{/Users/michael/Drive/thesis/figures/variance_over_m.png}

\caption{Panel (a): The variance in the change in abundance at each timestep, $V\Delta$, plotted across $p_m$. Each point represents $50$ replicates at that parameter set. Different colors indicate different levels of demographic stochasticity, $\sigma_p$. Panel (b): The value of $V(\Delta)$ (y-axis) and $PCC$ (x-axis) for each set of $50$ replicates. Different colors indicate different levels of demographic stochasticity, $\sigma_p$.  }

\label{variance_over_m}

\end{figure}

\begin{figure}[h]

\includegraphics[width=15cm]{/Users/michael/Drive/thesis/figures/delta_pcc_corr.png}

\caption{The value of $p_m$ which minimizes the variance of $\Delta$, $V(\Delta)$ (x-axis) and the value of $p_m$ which maximizes $PCC$, (y-axis). Each point represents the value of $p_m$ which maximizes synchrony, $m_{max}$,  across $p_m = \{0.01, 0.02, \dots, 1\}$ , at different $N_p=2,3,\dots,25$. The black line is $y=x$.}

\label{delta_pcc_corr}

\end{figure}

We know from theory that reducing that variance of \(N_t - N_{t-1}\)
reduces extinction risk (@lande\_stochastic\_2003,
@melbourne\_extinction\_2008). It has been suggested in the past the
spatial synchrony can increase the risk of extinction
(@matter\_local\_2010, @heino\_synchronous\_1997). However, our evidence
here runs counter to that claim---increasing synchrony decreases the
variance of the change in abundance over time, \(V(\Delta)\), which is
key in maximizing the probability of a populations persistence
(@lande\_stochastic\_2003). We suggest that prior studies are flawed in
that they do not consider that the advent of synchrony reduces the
probability of catastrophes, or high variance in \(\Delta\).

\pagebreak

\hypertarget{the-balance-of-migration-and-demographic-stochasticity}{%
\subsection{The Balance of Migration and Demographic
Stochasticity}\label{the-balance-of-migration-and-demographic-stochasticity}}

The question remains---does this diffusion model accurately describe the
actual dynamics of dispersal in ecological systems? Dispersal is a
stochastic process, and diffusion assumes that each time step, the
dispersal between pops in the expected value of that process. Here we
present a method for using our representation of landscape connectivity
to simulate stochastic dispersal, and compare the stochastic dispersal
model to the diffusion model.

\hypertarget{stochastic-dispersal-model}{%
\subsubsection{Stochastic Dispersal
Model}\label{stochastic-dispersal-model}}

We use the same construction of landscape connectivity in the random
dispersal model as the diffusion model---that is that each individual
migrates from its original population with probability \(P(m)\), and the
conditional distribution of where migrations go, \(P(V_j|V_i)\).
However, rather than using fixed coefficients, \(\Phi_{ij}\), to
represent dispersal, each generation we draw the number of total
migrations from population \(i\), \(N_{mi}\), from a Binomial
distribution, \(N_{mi} \sim \text{Binomial}(N_i,p_m)\). Then, for each
migrant \(\{1,2,N_{mi}\}\), the population \(j\) the migrant goes is
drawn directly from the distribution \(P(V_j|V_i)\)

\hypertarget{diffusion-vs.-stochastic-model}{%
\subsubsection{Diffusion vs.~Stochastic
Model}\label{diffusion-vs.-stochastic-model}}

First we consider the difference in synchrony transitions across a
migration gradient between using stochastic migration (\emph{Stochastic
Dispersal Model} above) vs.~the diffusion model (equation
\ref{diffusion_model}). Figure \ref{stoch_vs_determ_grid} shows phase
transition curves for both the diffusion (dark blue) and stochastic
(light blue) models across different values of \(N_p\) and \(\sigma_p\).

\begin{figure}[h]

\includegraphics[width=15cm]{/Users/michael/Drive/thesis/figures/stoch_vs_determ_grid.png}

\caption{Each panel shows a phase transition curves across $p_m$. The top row has $N_p=2$ and the bottom row has $N_p=5$. Columns correspond to different values of demographic stochasticity, $\sigma_p$. }

\label{stoch_vs_determ_grid}

\end{figure}

We see that as of demographic stochasticity, \(\sigma_p\), increases,
the difference in PCC between the deterministic model and the stochastic
model decreases. This is evident of a transition in which stochastic
mechanism, demographic stochasticity or dispersal stochasticity, is
driving the variation between populations at any fixed time.

In cases where the amount of demographic stochasticity is low (Figure
\ref{stoch_vs_determ_grid}A), migration events are the primary source of
the variability in the size of each population over time. When we model
dispersal deterministically, there is no variability in migration each
timestep. So, there is more correlation across populations in the
deterministic than the stochastic model, where migration events occur
according to Monte Carlo sampling. However, as we increase the amount of
demographic stochasticity, \(\sigma_p\), we reach a point where the
variability in the abundance of each population over time is dominated
by demographic stochasticity within populations, as opposed to migration
events. When demographic stochasticity is the primary source of
variance, migration between populations, even if it is not distributed
exactly according to the dispersal potential at each time point, spreads
this stochasticity out across all populations. When migration
stochasticity is the primary source of variance, the abundances at each
population \(i\) are stable, except for the variance induced by
immigration and emigration events.

Figure \ref{stoch_vs_determ_surface} show that beyond some threshold
level of demographic stochasticity, both the deterministic and
stochastic dispersal model can be used to understand the effects of
dispersal and landscape structure on correlation between dynamics as the
temporal variance that would `break down' correlations between
abundances is dominated by demography rather than the stochastic nature
of migration. We can better understand this transitions using the phase
diagram in Figure \ref{stoch_vs_determ_surface}b.

\begin{figure}[h]

\includegraphics[width=15cm]{/Users/michael/Drive/thesis/figures/stoch_vs_determ_surface.png}

\caption{(a): Phase transition diagram across $\sigma_p$ (x-axis) and $p_m$ (y-axis). Each cell represents the difference in mean PCC across $200$ replicates of both the stochastic dispersal and diffusion model. Labeled contour lines are in white. (b): The same phase transition diagram as (a), with regimes labeled. Note the logarithmic scale on the x-axes on both panels.}

\label{stoch_vs_determ_surface}

\end{figure}

\pagebreak

Figure \ref{stoch_vs_determ_surface} shows different two regimes---one
where the stochastic and diffusion model make the same predictions, and
one where they don't, highlighted in figure
\ref{stoch_vs_determ_surface}b. This indicates that as long as
demogaphic stochasticity is larger than \(\sigma_p = 0.05\), the
diffusion model and stochastic dispersal model make the same predictions
about metapopulation synchrony. In the migration dominated regime,
building up high synchrony isn't possible at any \(p_m\). However, in
the `demographic stochasticity' region, the that stochastic migration
and diffusion models make the same prediction, and develop the synchrony
transitions we see in the previous section.

\hypertarget{the-mesopopulation}{%
\subsection{The Mesopopulation}\label{the-mesopopulation}}

What does it mean biologically that this is a continuous phase
transition? When we increase the strength of isolation-by-distance, the
system does not collapse from one unifed whole to \(N_p\)
subdivisions---instead, synchronous dynamics form within subsets of the
total set of populations. In figure \ref{mesopop} we see the same
metapopulation plotted across varying values of isolation-by-distance,
\(\alpha\).

\begin{figure}[H]

\includegraphics[width=15cm]{/Users/michael/Drive/thesis/figures/mesopop.png}

\caption{Each panel (top, middle, bottom) represents the same set of $N_p=5$ populations. On the left, the population is plotted in the $x-y$ plane, with edge opacity and width corresponding the value of the dispersal potential $P(V_j | V_i)$. On the right is a set of pairwise correlation coefficient plots. Here, each point represents the mean value of crosscorrelation between those specific populations across $50$ replicates. Each panel represents different values of isolation-by-distance, $\alpha$.  }

\label{mesopop}

\end{figure}

At \(\alpha=0\), where dispersal is uniform, each pair of populations
shows a synchrony transition diagram that is identical, and we could
qualitatively say that this system functions as a whole.

As we increase \(\alpha\), (figure \ref{mesopop}b) we see that
populations \(4\) and \(5\) form a system between themselves, where
synchrony between those two populations is maximized near \(p_m=0.5\),
while the synchrony values between populations \(4\) and \(5\) with
populations \(1\), \(2\), and \(3\) moves toward \(PCC =0\). Meanwhile,
populations \(1\), \(2\), and \(3\) form their own system with
synchronous dynamics maximized at \(p_m = 1-\frac{1}{3}= \frac{2}{3}\),
as we would expect in a \(3\)-population system.

At \(\alpha = 15\), figure \ref{mesopop}b, these dynamics solidify, and
populations \(4\) and \(5\) form a single, dual-synchronized system, and
\(1\), \(2\) and \(3\) form a single \(3\)-population system.
Understanding how shifts in landscape connectivity cause whole system of
populations to break down into several mesopopulations, systems of that
form a unity through synchronized dynamics within the whole, has
important applied context. In some conservation scenarios, developing
corridors to connect these clusters could have significant benefits to
biodiversity and ecosystem functioning.

\hypertarget{discussion}{%
\subsection{Discussion}\label{discussion}}

Here, we present a way of measuring of \emph{functional connectivity}
using simulation models. We construct a model of diffusion that relates
to measurable values of resistance in real landscapes, and show that
this diffusion model can make accurate predictions about stochastic
dispersal at sufficiently high values of demographic stochasticity,
\(\sigma_p\).

We emphasize the importance of these types of simulation models in
measuring \emph{functional} connectivity. Given the constraints that
exist on conservation efforts, it is clear that we must make use of
limited resources in the most effective way possible if we want to
preserve biodiversity as it currently exists on Earth. Developing
corridors between existing habitat patches has seen extensive use as a
strategy in conservation (@beier\_conceptualizing\_2012). Within this
framework, the question still remains---what corridors are the most
effective to add? The resources available to conservation are often
tightly limited and as a result being able to make the most effective
decisions to preserve biodiversity given the potential resources
available is important. We propose that simulation models, like we
describe here, can be effective in assessing functional landscape
connectivity, and planning future landscape development such that humans
can live side-by-side with the rest of Earth's inhabitants.

\pagebreak
